\documentclass{article}
\usepackage{mymacro}
\lstset{language=c++}
\lstset{linewidth=9cm,frameround=tttt,basicstyle=\ttfamily,keywordstyle={}}

\parindent=0in
\parskip=.3cm

\def\C++{C{\raise 0.2ex\hbox{++}}}
\newcommand{\R}{$\bfR$}

\title{RCPPBIND: A \R/\C++ template class library}
\date{July 1, 2008}

\author{Gang Liang $<$gumpleon@gmail.com$>$}

%Dominick Samperi <dsamperi@DecisionSynergy.com>

\begin{document}
\maketitle

\begin{abstract}%
  \R\ is a powerful open-source statistical software, and
  the S language used in \R\ is flexible and easy to master
  for end-users. But unfortunately it is rather difficult for
  developers to write softwares for \R\ using languages
  such as \C++: under such circumstances, developers are
  forced to learn a great deal of internal structures and
  low-level operations of \R\ in order to manipulate \R\ objects.  The
  goal of this template \R/\C++ library, \texttt{rcppbind}, is to
  provide a simple interface between \C++\ and \R. It is the
  wish of the author that the burden on \R/\C++\ developers could be
  lessen in some degree.%
\end{abstract}

\section{Overview}

\subsection{Package Info}

The package \texttt{rcppbind} installs all \C++\ header files
and provides several demos on how to use the template
library. It only contains two \R\ functions:
\begin{enumerate}
  \item \texttt{rcppbind.demo}: some demonstration
    modules using the rcppbind template class library;
  \item \texttt{rcppbind.path}: return the path to the
    installed header files.
\end{enumerate}

\subsection{Library Overview}

Different from \C++ being a strong-type language, \R\ is of
weak type. In other words, all \R\ objects are handled
through an \textsl{SEXP} (simple expression) interface, which
is an opaque pointer. Each \R\ object has a class type, such
as \texttt{vector, matrix, list, data.frame}, and a data
type, such as \texttt{int, double, Rcomplex, char*, Date,
POSIXct}; roughly speaking, class types can be viewed as
wrappers for data types. Some examples of \R\ objects are a
vector of integers or a matrix of doubles. A simple but
important fact is that even a single integer in \R\ is indeed
a vector of integer of length 1.  R

The basic idea of \texttt{rcppbind} is to define a set of
classes to wrap \R\ objects in order to minimize the
effort of \R/\C++\ developers in mastering many low-level \R\
operations. Currently, the following four class types are
supported by the \texttt{rcppbind} template library:
\begin{center}
  \begin{tabular}{r|rrrrr} \hline\hline
    \R                & \texttt{vector} & \texttt{matrix} & \texttt{data.frame} & \texttt{list} \\
    \texttt{rcppbind} & \texttt{rvector} & \texttt{rmatrix} & \texttt{rdataframe} & \texttt{rlist} \\
    \hline\hline
  \end{tabular}
\end{center}
Among them, \texttt{rvector} and \texttt{rmatrix} are
template classes, while \texttt{rdataframe} and
\texttt{rlist} are regular classes. These classes serve as a
wrapper for fundamental data types. Details on how to wrap
\R\ objects can be found in Sec~\ref{sec:classtypes} and
Sec~\ref{sec:entrance-exit}.

Not all data types are born equal: some special data types
need special treatments, most noticeably \texttt{char*}. The
following \texttt{rcppbind} classes are designed to wrap
these outliers: 
\begin{center}
  \begin{tabular}{r|rrr} \hline\hline
    \R                & \texttt{char*} & \texttt{Date} & \texttt{POSIXct} \\
    \texttt{rcppbind} & \texttt{rstring} & \texttt{rdate} & \texttt{rtime} \\
    \hline\hline
  \end{tabular}
\end{center}
A detailed account of these special data types can be found
in the following sections.

\subsection{Design Issues}

In order to simplify the library design, the following
constraints or design specifications are imposed:
\begin{enumerate}
  \item The library only supports \texttt{.Call} interface in
    \R. This decision is made simply for convenience.
    Interested readers please refer to \R\ manual ``Writing
    \R\ Extensions''.
  
    As a consequence, the \texttt{rcppbind} library only
    supports the positional parameter-matching calling
    convention, which is natural for \C++. Since \R\
    also supports a name-matching calling convention: a good
    place to implement it is inside the \R\ function right
    before the \texttt{.Call} interface to the \C++\ module.
  \item In order to bind with \texttt{rcppbind} objects, one
    needs to know both class and data types of \R\ objects.
    The exception to this rule is that internal conversions
    will be performed for certain commonly convertible data
    types. For instance, a vector of integer in \R\ will be
    automatically converted to \texttt{rvector<double>}
    if such type is declared to wrap the input \R\ object. An
    exception will be thrown If the data type does not match
    and a conversion is not possible.
  \item In \R\, all objects should be protected before
    accessing the data area. This protect/unprotect mechanism
    is automatically enforced under the template library
    framework: a \texttt{rcppbind} object is protected
    when constructed, and unprotected when deleted. The
    library does not dictate the order of the object creation
    and deletion. In order to eliminate possible conflicts,
    the template library uses \texttt{UNPROTECT\_PTR} to
    unprotect objects, but it has a small performance penalty
    over \texttt{PROTECT}. Hence, it is recommended (but not
    strictly required) that the library users declare objects
    in a first-declare-last-delete order.
\end{enumerate}

The main reason people want to use such a template library is
to code some computational intensive modules by taking
advantage of efficiency of \C++\ over \R, and easiness of
\C++\ over C. It is author's presumption that the general
flow of the \R/\C++\ modules using a general template library
would be
\begin{enumerate}
  \item wrap \R\ input parameters;
  \item do operations using whatever \C++\ data structures
    and functions;
  \item return computed results to \R.
\end{enumerate}
The second step is out the scope of the simple manual, and
should refer to any \C++\ books. The details on how to wrap
and return \R\ objects can be found in
Sec~\ref{sec:entrance-exit}. Below we will first introduce
some \texttt{rcppbind} classes for \R\ class and data types.

\section{Class Types}\label{sec:classtypes}

In the current implementation, the template library only
supports four class types. Internally, \texttt{vector, list}
are the two most fundamental class types: all other class
types are derived from them. For instance, \texttt{matrix} is
internally \texttt{vector}, \texttt{data.frame} is
implemented through \texttt{list}.

\subsection{\texttt{rvector}}

The class template \texttt{rvector<T>} is designed to wrap
\R\ vector objects. The currently supported data types,
\texttt{T}, are \texttt{int, double, Rcomplex, char*, rdate,
rtime}.  Some of the most useful member functions of
\texttt{rvector<T>} are
\begin{enumerate}
  \item \texttt{rvector<T> (SEXP)}: constructor to wrap an
    existing \texttt{vector} object
  \item \texttt{rvector<T> (int)}: construct a new
    \texttt{rvector<T>}
  \item \texttt{int length()}: return the length of the
    vector
  \item \texttt{T* begin(), end()}: return the begin and end
    of the iterators of the vector
  \item \texttt{T\& operator [int], T\& operator [std::string]}: subset operator
  \item \texttt{operator =, *=, +=, -=}: (for numeric data
    types only)
\end{enumerate}

\subsubsection*{Notes}

\begin{enumerate}
  \item It is preferred to use iterators returned by
    \texttt{begin(), end()} to access members of a \R\ vector
    object. There is a small penalty of using
    \texttt{operator []}: the validity of the index will be
    checked when when each time called.
  \item The subset \texttt{operator []} is overloaded: it
    also supports searching elment by dimname -- a
    \texttt{std::string} object.
  \item \texttt{rvector<rstring>, rvector<rdate>,
    rvector<rtime>} are wrappers for special data types. See
    Sec~\ref{sec:datatypes} for more details.
\end{enumerate}

\subsection{\texttt{rmatrix}}

The class template \texttt{rmatrix<T>} is designed to wrap
\R\ matrix objects.  The currently supported data types
\texttt{T} are \texttt{int, double, Rcomplex, char*, rdate,
rtime}.  A list of important member functions of
\texttt{rmatrix<T>} is as follows:
\begin{enumerate}
  \item \texttt{rmatrix<T> (SEXP)}: constructor for an
    existing \R\ matrix
  \item \texttt{rmatrix<T> (int nrow, int ncol)}: constructor
  \item \texttt{int nrow(), ncol()}: return the dimension
    info of the object
  \item \texttt{T* begin(), end()}: the begin and end
    iterator of the whole data block
  \item \texttt{T* cbegin(int n), cend(int n)}: the begin and
    end iterators of the $n$th column;
  \item \texttt{T* cbegin(std::string), cend(std::string)}:
    the begin and end iterators of the column with the given
    name;
  \item \texttt{T\& operator (int nrow, int ncol)}: element access
  \item \texttt{operator =, *=, +=, -=} (for numeric data
    types only)
\end{enumerate}

\subsubsection*{Notes}

\begin{enumerate}
  \item Similar to \texttt{rvector}, it is preferred to use
    iterators returned by \texttt{cbegin, cend} to access
    column members of a \R\ matrix object: \R\ is a
    column-prior language.  The element access operator
    \texttt{()} would check the validity of the index each
    time when called.
  \item \texttt{rmatrix<rstring>, rmatrix<rdate>,
    rmatrix<rtime>} are wrappers for matrix of special data
    types. See Sec~\ref{sec:datatypes} for more details.
\end{enumerate}

\subsection{rdataframe}

The class \texttt{rdataframe} is designed to wrap \R\
\texttt{data.frame} objects, which are data tables of
rectangular shape. \texttt{rdataframe} is a regular class,
and we need to further bind each column of a
\texttt{rdataframe} object with a \texttt{rvector} object to
retrieve data from a data.frame object.  A list of important
member functions of \texttt{rdataframe} is as follows:
\begin{enumerate}
  \item \texttt{rdataframe(SEXP)}: constructor for wrapping
    an existing SEXP
  \item \texttt{rdataframe(int, int, std::vector<const
    std::type\_info*>\&, std::vector<std::string>\&)}: create
    a new data.frame object
  \item \texttt{rvector<T> getColumn(n)}: retrieve the $n$th
    column of the data.frame
  \item \texttt{rvector<T> getColumn(colname)}: retrieve the
    column with the given column name
  \item \texttt{SEXPTYPE getColType(n)}: return type of the $n$th column

  \item \texttt{int nrow(), ncol()}: return the dimension information
  \item \texttt{std::vector<std::string>\& getColNames()}:
    return the column names;
  \item \texttt{void setColNames(std::vector<std::string>\&)}: set column names
\end{enumerate}

\subsubsection*{Notes}
\begin{enumerate}
  \item Again, one needs to know both class and data types of
    a column in order to bind it with a \texttt{rvector}
    object. The usual syntax for applying the template member
    function \texttt{getColumn} is as follows:
    \begin{lstlisting}
    rdataframe df = ...
    rvector<double> datavec = df.getColumn<double>(0);
    \end{lstlisting}
  \item In order to construct a \texttt{rdataframe} object of
    interest, one needs to provide the dimension information,
    each column type info, plus column names. The column type
    info is denoted as \texttt{std::vector<const
    std::type\_info*}. The reason to use type info instead of
    \texttt{SEXPTYPE} is because it contains the complete information
    to determine the class type. See \texttt{demo\_df.cpp}
    for an example.
\end{enumerate}

\subsection{rlist}

The class \texttt{rlist} is designed to wrap \R\
\texttt{list} objects: \texttt{list} is the most flexible
class type in \R. A list of important member functions of
\texttt{rlist} is as follows:
\begin{enumerate}
  \item \texttt{rlist( SEXP sexp )}
  \item \texttt{rlist( int size, std::vector<std::string>
    strname = NULL )}
  \item \texttt{int size()}: return the size of the list
  \item \texttt{void setElement( int index, SEXP sexp )}
  \item \texttt{SEXP getElement( int  index )}
\end{enumerate}

\subsubsection*{Notes}
\begin{enumerate}
  \item The above \texttt{getColumn} and \texttt{setColumn}
    member functions can also take a \texttt{std::string}
    argument for searching elements by name.
  \item Both functions \texttt{getColumn} and
    \texttt{setColumn} takes or returns parameter of type
    \texttt{SEXP} because the underlying objects are unclear
    to the class itself.  Nevertheless, the \texttt{rcppbind}
    framework provides a general approach for handling these
    opaque \texttt{SEXP} pointers. Generally speaking,
    module writers know the types of these objects; hence, it
    is his/her responsibility to manually wrap these objects
    with the corresponding \texttt{rcppbind} classes. See
    \texttt{demo\_list.cpp} for a simple example.
\end{enumerate}

\section{Data Types}\label{sec:datatypes}

We discuss three special data types in this section. One
thing in common among these three data types is all these
wrapper classes for data types are served as intermediate to
access the underlying data members.

\subsection{rstring}

\texttt{char*} is a special data type which needs special
treatments because a character string itself is a
\textsl{SEXP} object in \R, so one more layer of indirection
has to be added to process \texttt{char*} data. The iterators
of \texttt{rvector<rstring>} and \texttt{rmatrix<rstring>}
are pointers to \texttt{rstring} object. Let \texttt{ptr} be
a valid iterator of \texttt{rvector<rstring>}. 
\begin{enumerate}
  \item The syntax to modify (or replace, to be precise) the
    character string being pointed is
    \begin{lstlisting}
    rvector<rstring>::iterator ptr = ...;
    *ptr = rstring("new string");
    \end{lstlisting}
  \item The syntax to extract the character string data from
    \texttt{ptr} is
    \begin{lstlisting}
    const char* str = ptr->c_str();
    \end{lstlisting}
    In other words, \texttt{rstring} is used as an
    intermediate to access elements of a string vector
    without knowing the underlying \R\ mechanism.
\end{enumerate}
See \texttt{demo\_str} for a simple example on how to
manipulate string characters.

\subsection{rdate}

\texttt{rdate} is wrapper for \R\ data type \texttt{"Date"}.
\R\ is a flexible system that a \texttt{"Date"} object could
be any numeric value (integer or real) as long as the value
is valid: it causes ambiguities for programmers.  Under the
\texttt{rcppbind} framework, the wrapper class \texttt{rdate}
is enforced to be integer values. A set of member functions
are provided to access the date information contained in the
given data; hence, the low-level implementation is hidden to
library users. A list of member functions is as follows:
\begin{enumerate}
  \item \texttt{time(struct tm\& tm)}: return the date info
    using a \texttt{tm} structure.
  \item \texttt{int operator-(const rdate\&)}: return the date difference;
  \item \texttt{rdate operator+(int)}: forward a date object;
  \item \texttt{static void date2tm(int date, struct tm\& tm),
    static int tm2date(const struct tm\& tm)}: date and
    \texttt{tm} conversion.
\end{enumerate}
 
\subsection{rtime}
    
\texttt{rtime} is wrapper for \R\ datatime type
\texttt{"POSIXct"}. Time is always a confusing issue in most
systems. In \R, there are two classes for time variables:
\texttt{POSIXlt, POSIXct}.
\begin{enumerate}
  \item The \texttt{POSIXct} (a calendar-time in POSIX
    standard) is a numeric value internally, while
    \texttt{POSIXlt} (a local-time in POSIX standard) is
    indeed a list with 9 fields.

    The benefit of \texttt{POSIXlt} is that all time
    information can be directly accessed from those 9 fields.
    Its disadvantage is that we cannot have a vector of
    \texttt{POSIXlt}: there is no such thing called a vector
    of list in \R. On the other hand, the compact form is
    just a numeric value (mostly real). We can put
    \texttt{POSIXct} in various data structures, but we need
    to compute time information from the numeric value.

    Indeed, the computational penalty brought by
    \texttt{POSIXct} is minimal.
  \item Many times, we see date-time object in \R\ with class
    attribute \texttt{POSIXt}. It is inherited from the above
    two classes to allow operations between \texttt{POSIXct}
    and \texttt{POSIXlt}.
\end{enumerate}

Under the \texttt{rcppbind} framework, the wrapper class
\texttt{rtime} is enforced to be real values -- natural for
\texttt{POSIXct}. A set of member functions are provided to
access the time information contained in the object. A list
of member functions is as follows:
\begin{enumerate}
  \item \texttt{void time(struct tm\& tm)} 
  \item \texttt{double operator-(const rtime\& obj)}: compute the time difference;
  \item \texttt{rtime operator+(double timediff)}: forward a time;
  \item \texttt{static void time2tm(double time, struct tm\&
    tm)}: time to \texttt{tm} structure;
  \item \texttt{static double tm2time(const struct tm\& tm)}:
    \texttt{tm} structure to time.
\end{enumerate}

\subsubsection*{Notes}
\begin{enumerate}
  \item The timezone information of \texttt{rtime} is not
    implemented: the default timezone is always assumed to be
    ``GMT''.
    
    Furthermore, consider a \R\ \texttt{POSIXct} object
    wrapped by \texttt{rvector<rtime>}. Timezone is an
    attribute of \texttt{vector} object, but not
    \texttt{rtime}. As an element of the datetime vector, the
    \texttt{rtime} is simply a real value internally.
\end{enumerate}

\section{Entrance \& Exit}\label{sec:entrance-exit}

\subsection{Entrance -- Retrieving Information from \R}

In order to minimize the programmers' burden in retrieving
\R\ objects from the \texttt{.Call} interface, a set of
rcppbind macros are defined to automatically wrap \R\ objects
into \C++\ objects under the \texttt{rcppbind} framework.
The syntax of the interface is as follows:
\begin{itemize}
  \item[] \texttt{RCPP\textsl{n}(\textsl{module\_name}, T1, ..., T\textsl{n})}
\end{itemize}
where \texttt{RCPP\textsl{n}} is a macro which defines a
\texttt{rcppbind} module, \texttt{\textsl{module\_name}},
with $n$ parameters, and \texttt{T1, ..., T\textsl{n}} are
the types of input arguments. The
\texttt{\textsl{module\_name}} is the entry point not only to the
\texttt{.Call} interface, but also to the function of real
implementation. One typical example is as follows:
\begin{lstlisting}
    #include <rcppbind.h>
    RCPP2(mymodule, double, rvector<double>);
    SEXP mymodule(const double& t1, const rvector<double>& t2) { ...; }
\end{lstlisting}
Inside \R, the \texttt{.Call} interface for calling this
module is as follows:
\begin{itemize}
  \item[] \texttt{mymodule <- function(x, y) .Call("mymodule", x, y)}
\end{itemize}
Of course, the module has to be loaded before the \R\
``\texttt{mymodule}'' function could be called. It is also a
good practice to ensure both \texttt{x, y} are of right types
inside the \R\ function \texttt{mymodule}.

\subsubsection*{Notes}
\begin{itemize}
  \item The current implementation only support 0 to 9 input
    arguments. If your module has more than 9 input
    parameters, please check the header file
    \texttt{rmacro.h} and redefine your own macro
    accordingly.
  \item The types \texttt{T1, \ldots, T\textsl{n}} in the
    macro definition can only be wrapper classes:
    \texttt{rvector, rmatrix, rdataframe, rlist}, and some
    primitive types, such as \texttt{int, Rcomplex, char*,
    rdate, rtime}.
  \item All input arguments are declared as references to
    constants: it conforms to the \R\ convention of parameter
    passing.
  \item The \C++\ exception mechanism is used in
    \textsl{module\_name}. If an error occurs during the
    execution of the module, the program will exit nicely and
    return the control to \R.
  
    Argument type checking is performed in the above macro to
    ensure that input arguments are indeed the right types
    declared in the \texttt{RCPP\textsl{n}} interface. An
    exception will be thrown if the parameter type does not
    match.
\end{itemize}

\subsection{Exit -- Returning Objects to \R}

The real implementation function is always of the form:
\begin{itemize}
  \item[] \texttt{SEXP \textsl{module\_name}(const ...);}
\end{itemize}
The return type of the function is always \texttt{SEXP}. It
is easy to do under the current framework.  In
\texttt{rcppbind}, all wrapper classes for class types,
namely \texttt{rvector, rmatrix, rlist, rdataframe}, are
derived from a common base-class \texttt{robject}, where an
operator \texttt{SEXP} is defined to convert them implicitly
to \texttt{SEXP}.

\subsubsection*{Notes}

\begin{itemize}
  \item If one wants to return an integer value to \R, one
    needs to return a vector of integer of length 1.  The
    same to \texttt{double, char*}, etc.
  \item For more complicated data structures, one has to rely
    on \texttt{rlist}. It is recommended to manipulate class
    attribute inside the \R\ function to the \texttt{.Call}
    interface.
\end{itemize}

\section{Summary}\label{sec:summary}

A quick summary of the \texttt{rcppbind} template library:
\begin{enumerate}
  \item The only header file needs to be included in your
    project is \texttt{rcppbind.h}.
  \item One has to know both class and data types of a \R\
    object in order to bind it with a \texttt{rcppbind}
    object.
  \item The module name in the \texttt{RCPP} macro is the
    entry point not only to the \texttt{.Call} interface,
    but also to the function of real implementation.
  \item The entry function only takes constant
    references as
    input arguments.
  \item The index of \C++\ starts at 0, while that of \R\
    starts at 1.
  \item Use \texttt{rstring} as an intermediate to access
    \texttt{char*} data.
\end{enumerate}

\end{document}
